\documentclass[preprint,12pt,authoryear]{LaTeXConfig/elsarticle} % Use elsarticle class

\input{LaTeXConfig/preamble.tex} % link to global preamble file
\input{LaTeXConfig/commands.tex} % link to global commands file. See LaTeXConfig/README.md
\begin{document}

% TODO: test fix <27-01-24> %
\title{The multi-objective Minkowski Sum Problem - Theory and definitions\\
%\vspace*{1.5ex}
\texorpdfstring{
\vspace*{1.5ex}
\small
% subtitle
%Generator sets and upper bounds.
}
}
\author[Aarhus University]{Mark Lyngesen}
\author[Aarhus University]{Sune Lauth Gadegaard}
\author[Aarhus University]{Lars Relund nielsen}
\affiliation[Aarhus University]{organization={CORAL - Cluster for OR Applications in Logistics, Department of Economics and Business Economics, BSS, Aarhus University},%Department and Organization
            addressline={Fuglsangs Allé 4}, 
            city={Aarhus},
            postcode={8000}, 
            state={Aarhus Municipality},
            country={Denmark}}

\begin{abstract}
%% Text of abstract
Some multi-objective optimization problems can be solved using decomposition methods where a set of smaller independent subproblems are solved. For this class of problems, the global objectives can be described as the sum of local objective values of the subproblems. We present some theoretical results for the nondominated sets of such problems and formulate so-called generator sets. The generator sets consists only of the necessarily nondominated points for each subproblem required for generating the global nondominated set. 
%Using the generator sets we define improved upper bound sets for the subproblems which only remove nondominated points which only contribute to global dominated solutions.
Using the generator sets we define improved upper bound sets that reduce the search area for nondominated (generator) points in the subproblems.
Using these generator upper bound sets we present a method of solving the otherwise independent subproblems in a way which solve the global problem by sharing information between subproblems.
\end{abstract}

\maketitle
 
\pagebreak

% TODO: write introduction <27-01-24> %
\section{Introduction}


\section{Literature review}

[Jeg vil foreslå at vi lægger dette i introduktionen. Se \cite{dietz2020introducing} for eksempel. Start bredt.]

Review of relevant literature

\begin{enumerate}
        \item Complex systems in generel
        \item Uncoupled systems
	\item Minkowski sum problems.
	\item Kerbérénès phd.
	\item filtering problems
\end{enumerate}



\section{Prerequisites}

\subsection{General MO}
\begin{itemize}
	\item Relations (sets and vectors)
	\item Bound sets.
	\item Operator notation: $\+ , \- , \msum... \times$
	\item Notation $\X, \Y, f, \X_E, \Yn...$
\end{itemize}


\section{Minkowski Sum Problem}
MSP definition for general p and S



\section{Theory}

\subsection{Theory for two subproblems S = {1,2}}

\subsection{Generator sets}
	
\subsection{Generator upper bound sets}
Consider different generator upper bound sets. $\Uc$ is generally defined by some known solution $\hYn^1 \subseteq \Yn^1$ and  $\hYn^1 \subseteq \Yn^2$. So far we have assumed $\hYn^2 = \Yn^2$.


\section{Emperical study}

The purpose of the computational study is to answer the following questions:

[General intro see e.g. https://www.research.relund.dk/publications/pdf/forget22b.pdf and https://www.research.relund.dk/publications/pdf/forget22a.pdf]

\begin{enumerate}
    \item How large is $|\Yn|$ given $|\Yn^s|, s\in S$ ($\prod_s |\Yn^s|)$?
    \item What is the relative size of the generator sets compared to $|\Yn^s|, s\in S$?
    \item Does the sequence of subproblems impact the generator set size? 
\end{enumerate}

\subsection{Test instances}

Possible setup: In an instance folder, create a subfolder for each instance = problem. Keep on csv file for each subproblem. Even better could be to have no subfolders but instead one json file for each instance. Another solution is to have a json file for each subproblem. This is properly the best option. However, the instances will be correlated. An instance can then be defined as a combination of subproblems.

For each subproblem we

\begin{itemize}
    \item Test $p=2,\ldots, 5$. [4 options]
    \item Width is the same for all objectives (\cite{Kerberenes2022phd} use $w_i=1000$ in Chap. 2) or width is the same for approx. half of the objectives (\eg 1000) and 1/4 for the remaining (\ie 250). [2 options]
    \item Generate on a sphere: either lower/west part (many supported) or upper/east part (many unsupported). [2 options]
    \item Generate $10, 50, 100, 150, 200$ points and find the nd points among them for each subproblem. [5 options]
    \item Classify each nd point (supported, extreme, unsupported)
    \item Keep statistics: number of points, supported, extreme, unsupported, ratio = supported/unsupported, (range) width $w_i$ = max-min, min and max value for each objective $i = 1,\ldots p$.
    \item 5 instances for each config. [5 options]
\end{itemize}

In total $4 \cdot 2 \cdot 2 \cdot 2 \cdot 5 \cdot 5 = 800$ files. Naming convention could be 

\verb|sub-<id>-<p>-<width>-<sphere-method>-<nd-points>-<ratio>-<instance-id>.json|.


Note that given integer points, an upper bound on the width of the master problem with $s = 1,\ldots, m$ subproblems is $w_i = \sum_s w_{si}$. That is, if we keep the width of the subproblem fixed, the width of the master grows as $m$ grows. 

Note that given integer points and $p=2$, an upper bound on the number of nd points is $1 + \min_i w_i$. For a problem with $m$ subproblems an upper bound on the nd points is $1 + \min_i \sum_p w_{pi}$

The following instance/problem groups are generated given combinations of (this should maybe be changed for each question):

\begin{itemize}
    \item $p=2,\ldots, 5$. [4 options]
    \item $m=2,\ldots 5$ where $m$ is the number of subproblems. [4 options]
    \item All subproblems have the same sphere config, half/half. [3 options]
    \item Increasing nd points in subproblems (if m=3 choose 100, 200, 250), all have highest (n=250). [2 options]
    \item 10 instances for each config. [10 options]
\end{itemize}

In total $4 \cdot 4 \cdot 3 \cdot 2 \cdot 10 = 960$ files. Naming convention could be 

\verb|prob-<id>-<p>-<m>-<sphere-method>-<sub-ids>.json|.

A result file in the format \href{https://github.com/MCDMSociety/MOrepo/blob/master/contribute.md}{MCDMSociety}   may be created.



\subsection{Emperical study of Generator sets}
When are generator sets small relative to the non-dominated sets?

\subsection{Emperical study of Generator upper bound sets}
When are upper bound sets from generators 'good'. (make precise)?


\subsection{Emperical study of information-sharing between subproblems}
***RQ? Formulate major research questions.

\begin{itemize}
	\item 
Generate a new point from a subproblem and update all generator sets and upper bound sets.
	\item 
Periodically generate a point for each subproblem whereafter all generator sets and upper bound sets are updated.
	\item 
Sweep: generate all supported non-dominated points for each subproblem whereafter all generator sets and upper bound sets are updated.
\end{itemize}



\subsection{Methods for answering research questions}

In the following we describe ways of generating test metrics used to answer the main empirical research questions. Each test is performed on an \msp instance given by a sequence of sets $\{\Y^s\}_{s \in \S}$ for some ordered sequence of subproblems indexed by a set $\S$ (where we assume $\Y^s \subseteq \Yn^s \forall s \in \S$).

\begin{enumerate}
	\item  How large is $\lvert \Yn \rvert$ given $\left\{\lvert \Yn^s \rvert \right\}_{s \in \S}$?
To determine the set $\Yn$ we run the naive Minkowski Sum filtering algorithm from \citep{Kerberenes2022}. This algorithm sequentially generates the nondominated Minkowski Sum and uses a nondominance filter as a subroutine. The speed of the nondominance algorithm differs for $p=2$ and for $p>2$.

In each test we record the following measures:
\begin{itemize}
	\item 
$\lvert \Yn \rvert$ total nondominated points.
	\item 
$\lvert \Ysn \rvert$ total supported nondominated points.
	\item 
$\lvert \Yen \rvert$ total extreme (supported) nondominated points.
	\item 
$\lvert \Yn \setminus \Ysn \rvert$ total unsupported nondominated points.
\end{itemize}



\item What is the relative size of the generator sets compared to $\lvert \Y^s \rvert (\forall s \in \S)$?
We first note that generator sets are not unique. We consider several different generator sets and compare. We define a \emph{Minimum generator set sequence} as a sequence of generators $\left\{ \Ycs  \right\}_{s \in \S}$ where $\Ycs \subseteq \Y^s$ such that $\bar{M} := \sum_{s \in \S} \lvert \Ycs \rvert$ is minimum.

Let $\bar{M}$ denote the size of a minimum generator set sequence. $\bar{M}$ is defined by the following combinatorial problem.

\begin{align}
	\bar{M}  =  \min& \sum_{s \in \S} \lvert \hY^s \rvert  \\
	s.t. \quad & \msum_{s \in \S} \lvert \hY^s \rvert = \Yn,  \label{constr:comb spanning}\\
		   & \hY^s \subseteq \Y^s, \quad \forall s \in \S
\end{align}

We will reformulate the combinatorial problem as a general MIP. To do this we introduce the following notation. We order the sets $\Y^s := \{y^s_1,\hdots\} = \{\Y_i^s\}_{s \in \I^s}$ indexed by set $\I^s$ for all subproblems $s \in s \in \S$. Likewise, we order the set $\Yn := \{y_j\}_{j \in \J}$ indexed by the set $\J$. We introduce decision variables $x$ and $w$ such that:
\begin{align}
	x^s_i &= \begin{cases}
    	1, &\text{ if }y_i^s \in \hY^s \\ 
    	0, &\text{ otherwise }\\ 
    \end{cases}
    & \forall i \in \I^s, \forall s \in \S
    \\
		w^s_{i,j} &= \begin{cases}
    	1, &\text{ if $y^s_i$ generates $y_j \in \Yn$} \\ 
    	0, &\text{ otherwise }\\ 
    \end{cases}
    & \forall i \in \I^s, \forall s \in \S, \forall j \in \J \nonumber
\end{align}
To enforce the definitions of the decision variables we add the following constraints.
\begin{align}
    w^s_{i,j} \le x^s_i, \quad \forall i \in \I^s, \forall s \in \S, \forall j \in \J \label{constr:MIP-subset-chosen}
\end{align}
That is, $y^s_i$ can only be part of generating $y_j$ (for any $j$) if $y^s \in \hY^s$. We now add constraints which ensure that each point in the set $\Yn$ is 'generated' by some combination of points of the subproblems:
\begin{align}
    \sum_{s \in \S}\sum_{i \in \I^s} w^s_{i,j} y^s_i = y_j, \quad \forall j \in \J \label{constr:MIP-generating}
\end{align}
Finally, each point $y \in \Yn$ must be a combination of exactly one point from each subproblem:
\begin{align}
    \sum_{i \in \I^s} w^s_{i,j} = 1, \quad \forall s \in \S, \forall j \in \J \label{constr:MIP-combination-single}
\end{align}
The constraints \Cref{constr:MIP-subset-chosen} together with \Cref{constr:MIP-combination-single} also ensures that at least one solution is chosen to each subproblem $\left( \sum_{i \in I^s} x^s_i \right) \forall s \in \S $.
The objective function is the sum of the chosen generator points: $\bar{M} := \sum_{s \in \S}\sum_{i \in \I^s} x^s_i$. This results in the following MIP:

\begin{align*}
	\bar{M}  := \min&  \sum_{s \in \S}\sum_{i \in \I^s} x^s_i \\
	s.t. \quad & w^s_{i,j} \le x^s_i & \forall i \in \I^s, \forall s \in \S, \forall j \in \J \\
		   &\sum_{s \in \S}\sum_{i \in \I^s} w^s_{i,j} y^s_i = y_j & \quad \forall j \in \J \\
		   &\sum_{i \in \I^s} w^s_{i,j} = 1 & \quad \forall s \in \S, \forall j \in \J  \\
		   & x^s_i \in \{0,1\} & \forall i \in \I^s, \forall s \in \S\\
		   & w^s_{i,j} \in \{0,1\} &\forall i \in \I^s, \forall s \in \S, \forall j \in \J \nonumber
\end{align*}
Given an optimal solution $(\bar{x},\bar{w})$ the corrospinding generator sets $\Ycs$ is gives as $\Ycs = \{ y^s_i \in \Y^s : \bar{x}^s_i = 1\}$.

For each test we record:
\begin{itemize}
	\item Total points, supported points, extreme (supported) points, unsupported nondominated points.
	\item Global $\lvert \Yn \rvert$, $\lvert \Ysn \rvert$,  $\lvert \Yen \rvert$, $\lvert \Yn \setminus \Ysn \rvert$.
	\item Subproblems (already known)  $\lvert \Yn^s \rvert$, $\lvert \Ysn^s \rvert$,  $\lvert \Yen^s \rvert$, $\lvert \Yn^s \setminus \Ysn^s \rvert$ $\forall s \in \S$.
	\item Generator sets $\lvert \Ycs \rvert$, $\lvert \Ycs \cap \Ysn^s \rvert$, $\lvert \Ycs \cap \Yen^s \rvert$,  $\lvert \Ycs \setminus \Ysn^s \rvert$  $\forall s \in \S$
	\item $\bar{M}$
\end{itemize}





\end{enumerate}


%\subsubsection{}



\input{sections/conclusion.tex}

Test cite
\citep{adelgren2018efficient}

\pagebreak

\bibliographystyle{LaTeXConfig/elsarticle-num-names} 
\bibliography{LaTeXConfig/literature}

\end{document}
