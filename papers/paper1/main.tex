\documentclass[preprint,12pt,authoryear]{LaTeXConfig/elsarticle} % Use elsarticle class

% document settings
\usepackage[utf8]{inputenc}
% language support
\usepackage[british]{babel}

% for hyperlinks
\usepackage{hyperref}

% for figures + tikz
\usepackage{tikz}
\usepackage{pgfplots}
\pgfplotsset{compat=1.18}
\usepackage{relsize}
\usepackage{graphicx}
\usepackage{caption}
\usepackage{subcaption}

% math stuff
\usepackage{amsmath,amsfonts,amssymb, mathtools}
%\mathtoolsset{showonlyrefs=true,showmanualtags=true} %Show only labels if make a reference to it
%\mathtoolsset{showonlyrefs}	
% numbered enviroments
\usepackage{amsthm}
\newtheorem{thm}{Theorem}[section]
\newtheorem{definition}{Definition}[section]
\newtheorem{corollary}{Corrolary}[section]
\newtheorem{remark}{Remark}[section]
\newtheorem{prop}{Proposition}[section]
\newtheorem{lemma}{Lemma}[section]
\newtheorem{exmp}{Example}[section]
\newtheorem{conj}{Conjecture}[section]

% for algorithms
\usepackage[linesnumbered,lined,boxed,commentsnumbered]{algorithm2e}

% for optimization problems
\usepackage[short]{optidef}

% for adding code
\usepackage{listings}

% for references
\usepackage{cleveref}
\crefname{subsection}{Subsection}{subsections}
\Crefname{prop}{Proposition}{proposition}
\Crefname{thm}{Theorem}{theorem}
\Crefname{definition}{Definition}{definition}
\Crefname{remark}{Remark}{remark}
\Crefname{lemma}{Lemma}{lemma}
\Crefname{exmp}{Example}{example}
\Crefname{conj}{Conjecture}{conjecture}


% for referencung optimization problems
\newcommand{\Pref}[1]{Problem \ref{#1}}
 % link to global preamble file
% # Abbriviations

% MO := Multi-objective
% MOCO := Multi-objective combinatorial optimization
% BOCO := Bi-objective combinatorial optimization
% MO-MSP := Multi-objective Minkowski sum optimization problem

% # Commands

% ## Ehrgott set notation
\newcommand{\X}{\mathcal{X}} % Decision space
\newcommand{\Xe}{\mathcal{X}_E} % Effecient solutions
\newcommand{\Y}{\mathcal{Y}} % Objective space points
%hide \newcommand[\Y][1][][\mathcal{Y}^{#1}] % objective space with argument
\newcommand{\Yn}{\mathcal{Y}_{\mathcal{N}}} % Non-dominated points
\renewcommand{\L}{\mathcal{L}} % Lower bound set
\newcommand{\U}{\mathcal{U}} % Upper bound set

% ## Article notation 

\newcommand{\mcN}{\mathcal{N}} % Non-dominated set
\newcommand{\YsS}{\bar{\mathcal{Y}}^s} % Conditioned non-dominated set
%hide \newcommand{\Gc}{\mathcal{G}^{1|2}} % set of generators
\newcommand{\Yc}{\bar{\mathcal{Y}}^{1}} % a minimal generator
\newcommand{\Ycs}{\bar{\mathcal{Y}}^{s}} % a minimal generator wrt. several subproblems.
\newcommand{\hY}{\hat{\mathcal{Y}}^{1}} % subset of non-dominated points
\newcommand{\hYn}{\hat{\mathcal{Y}}_\mathcal{N}} % subset of non-dominated points
%show \newcommand{\Yc}{\ ^g\bar{\mathcal{Y}}^{1}} % a minimal generator
%hide \newcommand{\Yc}{\bar{\prescript{g}{}{\mathcal{Y}}^{1}}} % a minimal generator with (g)
\newcommand{\hYc}{\hat{\mathcal{Y}}^{1}} % (another) minimal generator
\newcommand{\Uc}{\bar{\mathcal{U}}^{1}} % a conditional upper bound set
\newcommand{\bY}{\bar{\mathcal{Y}}} % bar over objective space
\newcommand{\bYn}{\bar{\mathcal{Y}}_\mathcal{N}} % bar over non-dominated points
\renewcommand{\u}{\bar{u}} % Auxilary variable used in upper bound algorithm
\newcommand{\yul}{y^{ul}} % Upper-left lex min solution
\newcommand{\ylr}{y^{lr}} % Lower-right lex min solution
\newcommand{\Q}{\mathcal{Q}} % Auxilary set used in upper bound algorithm


% ## index sets
\newcommand{\I}{\mathcal{I}} % index set of decision variables
\newcommand{\J}{\mathcal{J}} % index set
\renewcommand{\P}{\mathcal{P}} % index set objective space
\renewcommand{\S}{\mathcal{S}} % set of subsystems
\newcommand{\Z}{\mathcal{Z}} % set of subsystems

% ## Number sets
\newcommand{\R}{\mathbb{R}} % The real numbers
\newcommand{\Rp}{\mathbb{R}^p} % The positive cone in p-dimensions
\newcommand{\Rpp}{\mathbb{R}_{\geqq}^p} % The positive cone in p-dimensions including $0$
\newcommand{\Rppp}{\mathbb{R}_{\geq}^p} % The positive cone in p-dimensions without $0$
\newcommand{\N}{\mathbb{N}} % The set of natural numbes
%hide \newcommand{\Z}{\mathbb{Z}} % The set of integers

% ##  Operators
\newcommand{\+}{\oplus} % Minkowski sum
\renewcommand{\-}{\ominus} % Minkowski difference
\newcommand{\msum}{\bigoplus} % Minkowski sum (multiple)
\newcommand{\mmul}{\bigotimes} % Cartesian product (multiple)
\newcommand{\x}{\times} % Cartesian product
%show \renewcommand{}{\prod} % Cartesian product (multiple)
%show \newcommand{}{(\cdot)_\mathcal{N}} % Non-dominated points

% ##  Misc
\newcommand{\eg}{e.g.\@\xspace} %
\newcommand{\wrt}{wrt.\@\xspace} %
\newcommand{\ie}{i.e.\@\xspace} %
\newcommand{\BB}{branch-and-bound\@\xspace} %
\newcommand{\msp}{MO-MSP\@\xspace} %
\newcommand{\MSP}{multi-objective Minkowski Sum problem\@\xspace} %
\newcommand{\eproblem}{$\epsilon$-problem\@\xspace} %
\newcommand{\emethod}{$\epsilon$-method\@\xspace} %
 % link to global commands file. See LaTeXConfig/README.md
\begin{document}

% TODO: test fix <27-01-24> %
\title{The multi-objective Minkowski Sum Problem - Theory and definitions\\
%\vspace*{1.5ex}
\texorpdfstring{
\vspace*{1.5ex}
\small
% subtitle
%Generator sets and upper bounds.
}
}
\author[Aarhus University]{Mark Lyngesen}
\author[Aarhus University]{Sune Lauth Gadegaard}
\author[Aarhus University]{Lars Relund Nielsen}
\affiliation[Aarhus University]{organization={CORAL - Cluster for OR Applications in Logistics, Department of Economics and Business Economics, BSS, Aarhus University},%Department and Organization
            addressline={Fuglsangs Allé 4}, 
            city={Aarhus},
            postcode={8000}, 
            state={Aarhus Municipality},
            country={Denmark}}

\begin{abstract}
%% Text of abstract
Some multi-objective optimization problems can be solved using decomposition methods where a set of smaller independent subproblems are solved. For this class of problems, the global objectives can be described as the sum of local objective values of the subproblems. We present some theoretical results for the nondominated sets of such problems and formulate so-called generator sets. The generator sets consists only of the necessarily nondominated points for each subproblem required for generating the global nondominated set. 
%Using the generator sets we define improved upper bound sets for the subproblems which only remove nondominated points which only contribute to global dominated solutions.
Using the generator sets we define improved upper bound sets that reduce the search area for nondominated (generator) points in the subproblems.
Using these generator upper bound sets we present a method of solving the otherwise independent subproblems in a way which solve the global problem by sharing information between subproblems.
\end{abstract}

\maketitle
 
\pagebreak

% TODO: write introduction <27-01-24> %
\section{Introduction}


\section{Literature review}

Review of relevant literaure

\begin{enumerate}
	\item Minkowski sum problems.
	\item Kerbérénès phd.
	\item filtering problems
	\item Coupled problems.
\end{enumerate}



\section{Prerequisites/ methodology}

\subsection{General MO}
\begin{itemize}
	\item Relations (sets and vectors)
	\item Bound sets.
	\item Operator notation: $\+ , \- , \msum... \times$
	\item Notation $\X, \Y, f, \X_E, \Yn...$
\end{itemize}


\subsection{Minkowski Sum Problem}
MSP definition for general p and S



\section{Theory}

\subsection{Theory for two subproblems S = {1,2}}

\subsection{Generator sets}
	
\subsection{Generator upper bound sets}
Consider different generator upper bound sets. $\Uc$ is generally defined by some known solution $\hYn^1 \subseteq \Yn^1$ and  $\hYn^1 \subseteq \Yn^2$. So far we have assumed $\hYn^2 = \Yn^2$.


\section{Emperical study}
***RQ? Formulate major research questions.

Consider sweep methods. 
\begin{itemize}
	\item 
Generate a new point from a subproblem and update all generator sets and upper bound sets.
	\item 
Periodically generate a point for each subproblem whereafter all generator sets and upper bound sets are updated.
	\item 
Sweep: generate all supported non-dominated points for each subproblem whereafter all generator sets and upper bound sets are updated.
\end{itemize}


\subsection{Emperical study of Generator sets}
When are generator sets small relativt to the non-dominated sets.

\subsection{Emperical study of Generator upper bound sets}
When are upper bound sets from generators 'good'. (make precise).

	

\input{sections/conclusion.tex}

Test cite
\citep{adelgren2018efficient}

\pagebreak

\bibliographystyle{LaTeXConfig/elsarticle-num-names} 
\bibliography{LaTeXConfig/literature}

\end{document}
